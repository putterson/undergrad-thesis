\chapter{Introduction}
\section{Thesis Statement}
We aim to explore the effectiveness of the Multi-Index Hashing technique~\cite{norouzi2012fast} for the fast matching of large, binary, image descriptors in large databases for the purpose of fast, accurate image matching.

\section{Background: Problem and Related Work}
Image matching is an active area of research with wide ranging applications, but achieving good performance with respect to speed, space and accuracy is still a challenge. There has been growing interest in mapping image data onto compact binary codes for fast nearest neighbours search. Binary codes are storage efficient and comparisons require just a small number of machine instructions, such that millions of binary codes can be compared to another in less than a second. Binary codes can also be used as direct indices into hash tables which can provide constant time lookups. In spite of the efficiency and speed of using binary codes to represent image data, most image feature description is expressed as vectors of floating point numbers which must be compared using a measure of Euclidean distance. While there has been work on mapping these real valued descriptors to binary codes ~\cite{torralba2008small,norouzi2011minimal} the issue remains that these schemes provide an imperfect mapping from real valued vectors to binary codes, and that to achieve a strong correlation between the Euclidean distance for the real valued codes and the Hamming distance for the binary codes, large binary code lengths must be used. While matching long binary codes is still quite fast using linear scanning, they have posed a problem in that up until recently it was inefficient to quickly match binary codes longer than 30 bits ~\cite{torralba2008small} using the binary codes as indices into hash tables. With the introduction of the MIH technique it is possible to exactly match codes longer than \textasciitilde30 bits faster than linear scanning.

\subsection{Motivation}
The motivation behind this work is the ability for the MIH technique to enable the use of high performance binary descriptors such as Rotated BRIEF. Binary descriptors do have the advantage of being easy to compare because Hamming distance is easily computed and is efficient on computer systems, but current limitations significantly decrease the efficiency of using longer binary codes which can have better matching properties than popular Euclidean distance based descriptors such as SIFT and SURF ~\cite{rublee2011orb}. With the MIH method of calculating Hamming space nearest neighbours we gain performance in matching long binary codes. This performance gain is helpful for the hashing techniques on real valued descriptors, but also enables the use of natively generated binary descriptors which do not suffer from the degradation of matching rates (See Figure ~\ref{fig:hashmatching}) due to hashing from Euclidean to binary descriptors, are faster to generate, and allow better matching properties.

\subsection{Multi-Index Hashing}
The Multi-Index Hashing technique introduces a fast method of searching for exact k-nearest neighbours in hamming space which is also space efficient~\cite{norouzi2012fast} . Previous solutions were using binary codes as direct indices into hash tables but this approach was not used for codes longer than 32 bits because the memory requirements are prohibitive~\cite{torralba2008small}. While the direct hashing is fast for shorter binary codes, it is desirable to use longer codes as short codes do not preserve similarity as well.

In the research introducing the MIH technique, an emphasis is placed on the mathematical properties of the proposed data structure, but empirical analysis of the structure was carried out using 64 and 128 bit codes, generated using Locality Sensitive Hashing and Minimal Loss Hashing from SIFT and GIST descriptor datasets. Analysis of the performance properties of 256 bit codes was not included in the research so we sought to analyze the performance of MIH using 256 bit codes.